\documentclass[11pt]{report}

% Document dimensions
\usepackage{geometry}
\geometry{top=1.5cm, bottom=1.5cm, textwidth=15cm}

% Math related packges.
\usepackage{amsmath}
\usepackage{cancel}

% Natural Deduction package
\usepackage{proof}
\usepackage{mdframed}

% Experimenting to get long proofs to fit on a page.
\usepackage{pdflscape}
\usepackage{graphicx}
\usepackage{wrapfig}
\usepackage{lscape}
\usepackage{rotating}
\usepackage{epstopdf}


\newcommand{\temp}[2]{{\overline{#2}}^{#1}}

% Import the necessary preamble for the document. 
\usepackage{../../proofsPrograms}

% Fix the header space: start at the top of the page.
\usepackage{hyperref}


\begin{document}
	
	
% Heading for the tutorial	
\begin{center}
	{\bf MATH230: Tutorial Five [Solutions]}
\end{center}
\begin{center}
	{\bf Peano Arithmetic}
\end{center}


% Box with goals and relevent lecture notes.
\noindent\fbox{
	\parbox{\textwidth}{

		Key ideas
			\begin{itemize}
				\item Natural deduction practice, 
				\item Proofs using the identity rules of inference,
				\item Prove first-order sentences in theories of arithmetic,
				\item Use the induction schema of Peano arithmetic, and 
				\item Become exasperated enough to appreciate the help of proof assistants. 
			\end{itemize}

		Relevant Topic: Peano Arithmetic\\
		Relevant reading: \href{https://adam.math.hhu.de/\#/g/leanprover-community/nng4}{Natural Number Game}
		
	\vspace{0.2cm}

	Hand in exercises: 1c, 2a, 2b, 2c, and 5b\\ 
	{\bf Due Friday @ 5pm to the submission box on Learn.}
	}
}
% Discussion questions for tutor.
\newline
\vspace{0.5cm}

\noindent {\bf Discussion Questions}

	Proofs that make use of the axiom (schema) PA 7 

	$$[P(0) \land \forall x \ (P(x) \to P(s(x)))] \rightarrow \forall y (P(y))$$

	will prove statements of the form $\forall y \ P(y)$ with the use of modus ponens. This requires proving the antecedent conjunction: 

	$$[P(0) \land \forall x \ (P(x) \to P(s(x)))]$$

	This in turn requires proving each conjunct i.e. two proofs witnessing: 

	$$\PA \vdash P(0) \hspace{2cm} \PA \ \vdash \ \forall x \ (P(x) \to P(s(x)))$$

	If we piece these together, then we see that all proofs by induction have the form: 

	\begin{mdframed}
		\begin{center}
			$\begin{array}{c}
				\infer[\IND]{\forall y \ P(y)}
					{\infer[\land I]{[P(0) \land \forall x \ (P(x) \to P(s(x)))]}
						{\infer[]{P(0)}{\begin{array}{c} \vdots \\ \mathcal{D}_{BC}\end{array}}
						&
						\infer[\forall I]{\forall x \ (P(x) \to P(s(x)))}
							{\begin{array}{c} \vdots \\ \mathcal{D}_{IS}\end{array}}}}
			\end{array}$
		\end{center}
	\end{mdframed}

	\begin{enumerate}
		\item Identify the following steps involved in a proof by induction of the following sentence of Peano arithmetic: 
		
		$$\PA \ \vdash \ \forall x \ (0+x=x) $$

			\begin{enumerate}
				\item Identify the wff $P(x)$ to do induction on. 
				\item $\mathcal{D}_{BC}$: Write down the sequent $\PA \ \vdash \ P(0)$.
				\item $\mathcal{D}_{IS}$: Write down the sequent $\PA, P(n) \ \vdash \ P(s(n))$.	
			\end{enumerate}
	\end{enumerate}

% New page for tutorial exercises.
\newpage
{\bf Tutorial Exercises}
\begin{enumerate}

	\item Give natural deductions of the following theorems of identity.
	
	\begin{enumerate}
		\item $\vdash \ \forall x  \forall y \ x = y \to y = x$
		
		\begin{center}
			$\begin{array}{c}
				\infer[\forall I]{\forall x \ \forall y \ (x = y \to y = x)}
					{\infer[\to I,1]{a=b \to b=a}
						{\infer[=E]{b=a}
							{\temp{=I}{a=a}
							&
							\temp{1}{a=b}}}}
			\end{array}$
		\end{center}


		\item $\vdash \ \forall x \forall y \forall z \ (x = y \land y = z) \to x = z$
		
		{\bf Solution:}	To introduce the $\forall$ we argue from a general case. Furthermore, we are to prove an implication, so we add a temporary hypothesis to discharge later. So the crux of the proof comes down to showing: 

		$$(a=b)\land(b=c) \ \vdash \ a=c$$

			\begin{center}
				$\infer[\forall I]{\forall x \ \forall y \ \forall z \ (x=y \land y=z) \to x = z}
					{\infer[\to I,1]{(a=b \land b=c) \to a=c}
						{\infer[=E]{a = c}
							{\infer[\land E_{r}]{b = c}
							{\infer[1]{\cancel{a=b \land b=c}}
								{}}
							&
							\infer[\land E_{l}]{a = b}
								{\infer[1]{\cancel{a=b \land b=c}}
									{}}}}}$
			\end{center} 
	\end{enumerate}

	Parts (a) and (b) together with proofs from lectures show that identity is reflexive, symmetric, and transitive. Thus behaving like an equivalence relation - as one would hope of the definition of equals!
	
	\newpage
	\item In this question $\text{PA}$ denotes the first-order theory of Peano arithmetic which has signature PA$: \{0,s,+,\times\}$ and axioms:
	
	\begin{enumerate}
		\item[PA1] $\forall x \lnot(s(x) = 0)$
		\item[PA2] $\forall x \ \forall y ((s(x) = s(y)) \to (x = y))$
		\item[PA3] $\forall x \ (x + 0 = x)$
		\item[PA4] $\forall x \ \forall y \ (x + s(y) = s(x + y))$
		\item[PA5] $\forall x \ (x \times 0 = 0)$
		\item[PA6] $\forall x \ \forall y \ (x \times s(y) = (x \times y) + x)$
		\item[PA7] $[P(0) \land \forall x \ (P(x) \to P(s(x)))] \rightarrow \forall y (P(y))$
	\end{enumerate}	
	
	Provide deductions to prove the following sequents.

	{\bf Comments:} Axioms PA1 - PA6 are each of the form $\forall ...$ which means that $\forall E$ must be used to start these proofs. A lot of these proofs turn on the right choice of term to substitute in for $x,y,z$ in the $\forall E$ step. Look at the statement you're trying to prove and find the correct substitution into an axiom to make it come out. 

	Theorems involving the binary function $+$ will necessarily make use of PA3 and/or PA4 as those are the axioms defining the properties of addition. 

	Theorems involving the binary function $\times$ will necessarily make use of PA5 and/or PA6 as those are the axioms defining the properties of multiplication. 

	PA 1 is the only statement of the form ``this is not equal to that''. This means that showing any two things are not equal must ultimately boil down to showing that if they were equal, then (something like) 0=1 would follow.
	
	\begin{enumerate}
		\item $\text{PA} \vdash 1 + 1 = 2$
		
		{\bf Solution:} First the sequent should be desugared as 1,2 are not terms in PA. So the sequent in PA is the following: 

		$$ \PA \ \vdash \ s(0) + s(0) = s(s(0))$$

			\begin{center}
				$\begin{array}{c}
					\infer[=E]{s(0) + s(0) = s(s(0))}
						{\infer[]{s(0) + 0 = s(0)}
							{\PA 3}
						&
						\infer[\forall E]{s(0) + s(0) = s(s(0) + 0)}
							{\PA 4}}
				\end{array}$
			\end{center}

		\item $\text{PA} \vdash 3 \neq 1$
		
		{\bf Solution:} Desugaring into PA yields the following sequent:

		$$ \PA \ \vdash \ \lnot[s(s(s(0))) = s(0)]$$

		Recall that introducing a negation ultimately comes down to showing 

		$$ \PA, \ s(s(s(0))) = s(0) \ \vdash \ \bot$$

		and then using $\to I$ to tidy up at the end. 

			\begin{center}
				\footnotesize{$\begin{array}{c}
					\infer[\to I,1]{\lnot[s(s(s(0))) = s(0)]}
						{\infer[\MP]{\bot}
							{\infer[\MP]{s(s(0))=0}
							{\infer[1]{\cancel{s(s(s(0))) = s(0)}}{}
							&
							\infer[\forall E]{[s(s(s(0)))=s(0)] \to [s(s(0)) = 0]}
								{\PA 2}}
						&
						\infer[\forall E]{\lnot[s(s(0))=0]}
							{\PA 1}}}
				\end{array}$}
			\end{center}
		
		\item $\text{PA} \vdash \forall x \ (x + 1 = s(x))$
		
		{\bf Solution:} Desugaring into PA yields the following sequent:

		$$\PA \vdash \forall x \ [x + s(0) = s(x)]$$

		\begin{center}
			$\begin{array}{c}
				\infer[\forall I]{\forall x \ [x + s(0) = s(x)]}
					{\infer[=E]{a + s(0) = s(a)}
						{\infer[\forall E]{a + 0 = a}{\PA 3}
						&
						\infer[\forall E]{a + s(0) = s(a+0)}{\PA 4}}}
			\end{array}$
		\end{center}

		\item $\text{PA} \vdash \forall x \ (x \times 1 = x)$ 
		
		{\bf Solution:} Desugaring into PA yields the following sequent: 

		$$\PA \ \vdash \ \forall x \ [x \times s(0) = x] $$

		This proof will make use of Exercise 3a of this tutorial. 

			\begin{center}
				\footnotesize{$\begin{array}{c}
					\infer[\forall I]{\forall x \ [x \times s(0) = x]}
						{\infer[=E]{s(a) \times s(0) = s(a)}
							{\infer[]{s(a) \times s(0) = 0 + s(a)}
								{\infer[\forall E]{s(a) \times s(0) = (s(a) \times 0) + s(a)}{\PA 6}
								&
								\infer[\forall E]{s(a) \times 0 = 0}{\PA 5}}
							&
							\infer[\THM]{0 + s(a) = s(a)}{\PA}}}
				\end{array}$}
			\end{center}

	\end{enumerate}

	\newpage
	\item The first-order language of Peano Arithmetic is often presented with an extra binary relation symbol $<$ where $x < y$ is given the usual interpretation: $x$ is \emph{strictly} less than $y$. In fact it is not necessary to add anything extra, for this relation can be defined using a sentence in $\PA$ as stated. 
	
	Write down a wff in $\PA$ which defines the binary relation $<$ of being ``strictly less than''. Use this to write down formulae that represent: less than or equal to, strictly greater than, and greater than or equal to. 

	%$$ x < y :\equiv \exists z \ (x + s(z) = y) $$

	\item Write down well-formed formulae in the first-order language of $\PA$ corresponding to the following statements. 
	
	\begin{enumerate}
		\item Each natural number is either equal to $0$ or greater than $0$. 			
		\item If $x$ is not less than $y$, then $x$ equals $y$ or $y$ is less than $x$.
		\item If $x$ is less than or equal to $y$ and $y$ is less than or equal to $x$, then $x=y$. 
	\end{enumerate} 

	\newpage
	\item The followings sequents all require the use of the induction axiom schema. Recall that all proofs using the induction schema have the following form: 
	
		\begin{mdframed}
			\begin{center}
				$\begin{array}{c}
					\infer[\IND]{\forall y \ P(y)}
						{\infer[\land I]{[P(0) \land \forall x \ (P(x) \to P(s(x)))]}
							{\infer[]{P(0)}{\begin{array}{c} \vdots \\ \mathcal{D}_{BC}\end{array}}
							&
							\infer[]{\forall x \ (P(x) \to P(s(x)))}
								{\begin{array}{c} \vdots \\ \mathcal{D}_{IS}\end{array}}}}
				\end{array}$
			\end{center}
		\end{mdframed}

	For this reason, once the wff $P(x)$ is identified, it suffices to provide the base case deduction $\mathcal{D}_{BC}$ and induction step $\mathcal{D}_{IS}$. The sequents are stated in such a way as to mean induction on the variable $x$ will be the easiest approach. Always do induction on the variable $x$.  
	
	\begin{enumerate}
		\item $\text{PA} \vdash \forall x \ (0 + x = x)$
		
		{\bf Solution:} 

		We prove this using induction on the wff 

		$$ P(x) : \ (0 + x = x)$$

		$\mathcal{D}_{BC}$: First state and prove the base case 

		$$ \PA \vdash (0 + 0 = 0)$$

		\begin{mdframed}
			\begin{center}
				$\begin{array}{c}
					\infer[\forall E]{0 + 0 = 0}
						{\PA 3}
				\end{array}$
			\end{center}
		\end{mdframed}

		$\mathcal{D}_{IS}$: Next state and prove the induction step

		$$ \PA, \ (0 + n = n) \ \vdash \ (0 + s(n) = s(n)) $$

		\begin{mdframed}
			\begin{center}
				$\begin{array}{c}
					\infer[\to I, \IH]{(0 + n = n) \to (0 + s(n) = s(n))}
						{\infer[=E]{0 + s(n) = s(n)}
							{\infer[\forall E]{0 + s(n) = s(0+n)}{\PA 4}
							&
							\infer[\IH]{\cancel{0+n=n}}{}}}
				\end{array}$
			\end{center}
		\end{mdframed}
		
		\newpage
		\item $\text{PA} \vdash \forall x \ (0 \times x = 0)$


		{\bf Solution:} 

		This proof is by induction on the wff 

		$$ P(x) : \ (0 \times x = 0)$$

		$\mathcal{D}_{BC}$: First state and prove the base case 

		$$ \PA \vdash 0 \times 0 = 0$$

		\begin{mdframed}
			\begin{center}
				$\begin{array}{c}
					\infer[\forall E]{0 \times 0 = 0}
						{\PA 5}
				\end{array}$
			\end{center}
		\end{mdframed}

		$\mathcal{D}_{IS}$: Next state and prove the induction step

		$$ \PA, \ 0 \times n = 0 \ \vdash \ 0 \times s(n) = 0$$

		\begin{mdframed}
			\begin{center}
				$\begin{array}{c}
					\infer[\to I, \IH]{(0 \times n = 0) \to (0 \times s(n) = 0)}
						{\infer[=E]{0 \times s(n) = 0}
							{\infer[=E]{0 \times s(n) = 0 \times n}
								{\infer[\forall E]{0 \times s(n) = 0 \times n + 0}{\PA 6}
								&
								\infer[\forall E]{0 \times n + 0 = 0\times n}{\PA 3}}
							&
							\infer[\IH]{\cancel{0 \times n = 0}}{}}}
				\end{array}$
			\end{center}
		\end{mdframed}
		
		\newpage
		\item $\text{PA} \vdash \forall x \ (1 \times x = x)$

		{\bf Solution:} 

		This proof is by induction on the wff 

		$$ P(x) : \ s(0) \times x = x$$

		$\mathcal{D}_{BC}$: First state and prove the base case 

		$$ \PA \vdash s(0) \times 0 = 0$$

		\begin{mdframed}
			\begin{center}
				$\begin{array}{c}
					\infer[\forall E]{s(0)\times 0 = 0}
						{\PA 5}
				\end{array}$
			\end{center}
		\end{mdframed}

		$\mathcal{D}_{IS}$: Next state and prove the induction step

		$$ \PA, \ s(0)\times n = n \ \vdash \ s(0)\times s(n) = s(n)$$

		The deduction below makes use of an exercise above. 

		\begin{mdframed}
			\begin{center}
				\footnotesize{$\begin{array}{c}
					\infer[\to I, \IH]{(s(0)\times n = n) \to (s(0)\times s(n) = s(n))}
						{\infer[=E]{s(0)\times s(n) = s(n)}
							{\infer[=E]{s(0)\times s(n) = n + s(0)}
								{\infer[\forall E]{s(0)\times s(n)= (s(0)\times n) + s(0)}{\PA 6}
								&
								\infer[\IH]{\cancel{s(0)\times n = n}}{}}
							&
							\infer[\THM]{n + s(0) = s(n)}{\PA}}}
				\end{array}$}
			\end{center}
		\end{mdframed}
		
		\newpage
		\item $\text{PA} \vdash \forall x \ (x = 0 \lor \exists y (x = s(y)))$ \hfill (Challenge!)

		{\bf Solution:} 

		This proof is by induction on the wff 

		$$ P(x) : \ [x = 0 \lor \exists y (x = s(y))]$$

		$\mathcal{D}_{BC}$: First state and prove the base case 

		$$ \PA \vdash [0 = 0 \lor \exists y (0 = s(y))]$$

		\begin{mdframed}
			\begin{center}
				$\begin{array}{c}
					\infer[\lor I]{0=0 \lor \exists y \ (0 = s(y))}
						{\infer[=I]{0=0}	
							{}}
				\end{array}$
			\end{center}
		\end{mdframed}

		
		$\mathcal{D}_{IS}$: Next state and prove the induction step

		$$ \PA, \ [n = 0 \lor \exists y (n = s(y))] \ \vdash \ [s(n) = 0 \lor \exists y (s(n) = s(y))]$$

		Since the induction hypothesis is a disjunction, the proof will finish with a disjunction elimination step. This requires proving the following sequents along the way: 

		$$ \PA \ \vdash \ (n=0) \to [s(n) = 0 \lor \exists y (s(n) = s(y))]$$

		\begin{mdframed}
			\begin{center}
				$\begin{array}{c}
					\infer[\to I,1]{(n=0) \to [(s(n)=0) \lor \exists y \ (s(n)=s(y))]}
								{\infer[\lor I]{(s(n)=0) \lor \exists y \ (s(n)=s(y))}
									{\infer[\exists I]{\exists y \ (s(n)=s(y))}
										{\infer[\MP]{s(n)=s(0)}
											{\infer[\THM]{n=0 \to s(n)=s(0)}{}
											&
											\infer[1]{\cancel{n=0}}{}}}}}
				\end{array}$
			\end{center}
		\end{mdframed}

$$ \PA \ \vdash \ \exists y \ (n=s(y)) \to [s(n) = 0 \lor \exists y (s(n) = s(y))]$$

		\begin{mdframed}
			\begin{center}
				$\begin{array}{c}
					\infer[\to I,2]{\exists y \ (n=s(y)) \to [(s(n)=0) \lor \exists y \ (s(n)=s(y))]}
								{\infer[\exists E]{(s(n)=0) \lor \exists y \ (s(n)=s(y))}
									{\infer[2]{\cancel{\exists y \ (n=s(y))}}{}
									&
									\infer[\to I,3]{n=s(w) \to [(s(n)=0) \lor \exists y \ (s(n)=s(y))]}
										{\infer[\lor I]{(s(n)=0) \lor \exists y \ (s(n)=s(y))}
											{\infer[\exists I]{\exists y \ (s(n)=s(y))}
												{\infer[\MP]{s(n)=s(s(w))}
													{\infer[3]{\cancel{n=s(w)}}{}
													&
													\infer[\THM]{(n=s(w)) \to (s(n)=s(s(w)))}
														{}
													}}}}}}
				\end{array}$
			\end{center}
		\end{mdframed}

		See over the page for these steps combined with the disjunction elimination to complete the proof of the induction step. 
		
		\begin{sidewaysfigure}

			$$ \PA, \ [n = 0 \lor \exists y (n = s(y))] \ \vdash \ [s(n) = 0 \lor \exists y (s(n) = s(y))]$$

			\begin{mdframed}
				\begin{center}
					\small{$\begin{array}{c}
						\infer[\lor E]{[(s(n)=0) \lor \exists y \ (s(n)=s(y))]}
							{\infer[\IH]{n=0 \lor \exists y \ (n = s(y))}{}
							&
							\infer[\to I,1]{(n=0) \to [(s(n)=0) \lor \exists y \ (s(n)=s(y))]}
								{\infer[\lor I]{(s(n)=0) \lor \exists y \ (s(n)=s(y))}
									{\infer[\exists I]{\exists y \ (s(n)=s(y))}
										{\infer[\MP]{s(n)=s(0)}
											{\infer[\THM]{n=0 \to s(n)=s(0)}{}
											&
											\infer[1]{\cancel{n=0}}{}}}}}
							&
							\infer[\to I,2]{\exists y \ (n=s(y)) \to [(s(n)=0) \lor \exists y \ (s(n)=s(y))]}
								{\infer[\exists E]{(s(n)=0) \lor \exists y \ (s(n)=s(y))}
									{\infer[2]{\cancel{\exists y \ (n=s(y))}}{}
									&
									\infer[\to I,3]{n=s(w) \to [(s(n)=0) \lor \exists y \ (s(n)=s(y))]}
										{\infer[\lor I]{(s(n)=0) \lor \exists y \ (s(n)=s(y))}
											{\infer[\exists I]{\exists y \ (s(n)=s(y))}
												{\infer[\MP]{s(n)=s(s(w))}
													{\infer[3]{\cancel{n=s(w)}}{}
													&
													\infer[\THM]{(n=s(w)) \to (s(n)=s(s(w)))}
														{}
													}}}}}}}
					\end{array}$}
				\end{center}
			\end{mdframed}
		\end{sidewaysfigure}		
		
		\newpage
		\item $\text{PA} \vdash \forall x \ \forall y \ [s(y) + x = s(y+x)]$ \hfill (Challenge!)

		{\bf Solution:} 

		This proof is by induction on the wff 

		$$ P(x) : \ \forall y \ [s(y) + x = s(y+x)]$$

		$\mathcal{D}_{BC}$: First state and prove the base case 

		$$ \PA \vdash \forall y \ [s(y) + 0 = s(y+0)]$$

		\begin{mdframed}
			\begin{center}
				$\begin{array}{c}
					\infer[=E]{s(a) + 0 = s(a + 0)}
						{\infer[=E]{s(a) + 0 = s(a)}
							{\infer[=I]{s(a)=s(a)}{}
							&
							\infer[\forall E]{s(a)+0=s(a)}{\PA 3}}
						&
						\infer[\forall E]{a = a + 0}{\PA 3}}
				\end{array}$
			\end{center}
		\end{mdframed}

		$\mathcal{D}_{IS}$: Next state and prove the induction step

		$$ \PA, \ \forall y \ [s(y) + n = s(y+n)] \ \vdash \ \forall y \ [s(y) + s(n) = s(y+s(n))]$$

		\begin{mdframed}
			\begin{center}
				\footnotesize{$\begin{array}{c}
					\infer[=E]{s(a)+s(n) = s(a + s(n))}
						{\infer[]{s(a)+s(n)=s(s(a+n))}
							{\infer[\forall E]{s(a)+s(n)=s(s(a)+n)}{\PA 4}
							&
							\infer[\IH]{s(a) + n = s(a+n)}{}}
						&
						\infer[\forall E]{a + s(n) = s(a + n)}{\PA 4}}
				\end{array}$}
			\end{center}
		\end{mdframed}
		
		These two proofs can be pieced together with the induction rule of inference to prove the original goal. 
		
		\newpage
		\item $\text{PA} \vdash \forall x \ \forall y \ \forall z \ [(y + z) + x = y + (z + x)]$ \hfill (Challenge!)
		
		This proof is by induction on the formula 
		
		$$P(x)  \ : \ \forall y \ \forall z \ [(y + z) + x = y + (z + x)] $$
		
		$\mathcal{D}_{BC}$: First state and prove the base case 
		
		$$\PA \ \vdash \  \forall y \ \forall z \ [(y + z) + 0 = y + (z + 0)] $$
		
		\begin{mdframed}
			\begin{center}
				$\begin{array}{c}
					\infer[\forall I]{\forall y \ \forall z \ [(y + z) + 0 = y + (z + 0)]}
						{\infer[=E]{(b + c) + 0 = b + (c + 0)}
							{\infer[\forall E]{(b+c)+0 = b+c}
								{\PA 3}
							&
							\infer[\forall E]{c + 0 = c}
								{\PA 3}}}
				\end{array}$
			\end{center}
		\end{mdframed}
		
		See the next page for induction step. 

		\begin{sidewaysfigure}
		$\mathcal{D}_{IS}$: Next state and prove the induction step

		$$ \PA, \ \forall y \ \forall z \ [(y + z) + n = y + (z + n)] \ \vdash \ \forall y \ \forall z \ [(y + z) + s(n) = y + (z + s(n))] $$
		
		%\newpage
		
			\begin{mdframed}
				\begin{center}		
					$\begin{array}{c}
						\infer[=E]{(b + c) + s(n) = b + (c + s(n))}
							{\infer[=E]{(b + c) + s(n) = b + s(c + n)}
								{\infer[=E]{(b+c)+s(n)=s(b+(c+n))}
									{\infer[\forall E]{(b + c) + s(n) = s((b+c)+n)}
										{\PA 4}
									&
									\infer[\IH]{(b+c)+n = b + (c+n)}{}}
								&
								\infer[\forall E]{b + s(c + n) = s(b + (c +n))}
									{\PA 4}}
							&
							\infer[\forall E]{c + s(n) = s (c + n)}{\PA 4}}
					\end{array}$
				\end{center}
			\end{mdframed}
		\end{sidewaysfigure}
		
		\newpage
		\item $\text{PA} \vdash \forall x \ \forall y \ [y + x = x + y]$ \hfill (Challenge!)
		
		{\bf Solution:} 

		This proof is by induction on the wff 

		$$ P(x) : \ \forall y \ [y + x  = x + y]$$

		$\mathcal{D}_{BC}$: First state and prove the base case 

		$$ \PA \vdash \forall y \ (y + 0 = 0 + y)$$

		\begin{mdframed}
			\begin{center}
				$\begin{array}{c}
					\infer[=E]{a + 0 = 0 + a}
						{\infer[=E]{a + 0 = a}
							{\infer[=I]{a=a}{}
							&
							\infer[\forall E]{a + 0 = a}{\PA 3}}
						&
						\infer[\THM]{0 + a = a}{\PA}}
				\end{array}$
			\end{center}
		\end{mdframed}

		$\mathcal{D}_{IS}$: Next state and prove the induction step

		$$ \PA, \ \forall y \ (y + n  = n + y) \ \vdash \ \forall y \ (y + s(n)  = s(n) + y)$$

		\begin{mdframed}
			\begin{center}
				$\begin{array}{c}
					\infer[=E]{a + s(n) = s(n) + a}
						{\infer[=E]{a + s(n) = s(n + a)}
							{\infer[\forall E]{a + s(n) = s(a+n)}
								{\PA 4}
							&
							\infer[\IH]{a + n = n + a}
								{}}
						&
						\infer[\THM]{s(n) + a = s(n+a)}{\PA}}
				\end{array}$
			\end{center}
		\end{mdframed}
		
		See the Logic section on the course webpage for a complete proof of the commutativity of addition, if you dare.
		
		\newpage
	\end{enumerate}

	\newpage
	\item Provide natural deductions of the following theorems of Peano Arithmetic. Beware each of these must be true, but some I have not provided natural deductions for. Some may require breaking down into further subgoals (lemma) to help. I recommend writing informal proofs, before formalising them with natural deductions.
	
		\begin{enumerate}
			\item $\PA , \ 0 < a \ \vdash \ 0 < s(a)$
			
			Recall that $0 < a$ and $0 < s(a)$ are ($\exists$) existential claims:
			
			$$0 < a := \ \exists x \ a = 0 + s(x)$$
			$$0 < s(a) := \ \exists x \ s(a) = 0 + s(x)$$
			
			As such, we must use an instance of $\exists$ elimination to make use of such an hypothesis.
			
			\begin{center}
				$\begin{array}{c}
					\infer[\exists E]{0 < s(a)}
						{0 < a
						&
						\infer[\to I,1]{a = 0 + s(a) \to 0 < s(a)}{
							\infer[\exists I]{0 < s(a)}
								{\infer[=E]{s(a) = 0 + s(s(t))}
									{\infer[\forall E]{s(s(t)) = 0 + s(s(t))}
										{\infer[\THM]{\forall x \ x = 0 + x}{}}
									&
									\infer[=E]{s(a) = s(s(t))}
										{\infer[\text{CONG}]{s(a) = s(0 + s(t))}
											{\temp{1}{a = 0 + s(a)}}
										&
										\infer[\forall E]{s(t) = 0 + s(t)}
											{\infer[\THM]{\forall x \ x = 0 + x}{}}}}}}}
				\end{array}$			
			\end{center}
			
			
			%\newpage
			\item $\PA \ \vdash \ a < s(a)$
			\item $\PA , \ a < b \ \vdash \ s(a) < s(b)$
			
			$$a < b := \exists x \ b = a + s(x)$$
			$$s(a) < s(b) := \exists x \ s(b) = s(a) + s(x)$$
			
			\begin{center}
				$\begin{array}{c}
					\infer[\exists E]{s(a) < s(b)}
						{a < b
						&
						\infer[\to I,1]{b = a + s(t) \to s(a) < s(b)}
						{\infer[\exists I]{s(a) < s(b)}
							{\infer[=E]{s(b) = s(a) + s(t)}
								{\infer[\text{CONG}]{s(b) = s(a + s(t))}
									{\temp{1}{b = a + s(t)}}
								&
								\infer[\forall E]{s(a) + s(t) = s(a + s(t))}
									{\infer[\THM]{\forall x \ \forall y \ s(x) + y = s(x + y)}
										{}}}}}}				
				\end{array}$
			\end{center}
			
			\item $\PA , \ s(a) < s(b) \ \vdash \ a < b$
			
			\newpage
			\item $\PA , \ (a < b) \land (b < c) \ \vdash \ a < c$
			{\scriptsize
			\begin{center}
				$\begin{array}{c}
					\infer[\exists E]{a < c}
						{\infer[\land E_{L}]{a < b}
							{(a < b) \land (b < c)}
						&
						\infer[\to I,1]{a < b \to c < a}
							{\infer[\exists E]{c < a}
								{\infer[\land E_{R}]{b < c}{(a < c) \land (b < c)}
								&
								\infer[\to I,2]{b < c \to a < c}
									{\infer[\exists I]{a < c}
										{\infer[=E]{c = a + s(x + s(x))}
											{\infer[\text{ASSOC}]{c = a + (s(x) + s(x))}
												{\infer[=E]{c = (a + s(x)) + s(x)}
													{\temp{1}{b = a + s(x)}
													&
													\temp{2}{c = b + s(x)}}}
											&
											\infer[\forall E]{s(x) + s(x) = s(x + s(x))}{\PA 4}}}}}}}
				
				\end{array}$			
			\end{center}}
			
			%\newpage
			\item $\PA \ \vdash \ \forall x \ [(x = 0) \lor (0 < x)] \hfill \text{(Challenge!)}$
			
			This can be proved by splitting into cases according to the following theorem:
			
			$$\PA \ \vdash \ x = 0 \lor \exists y \ x = s(y)$$
			
			Thus we have the subgoals: 			
			$$\PA, \ t = 0 \ \vdash \ t = 0 \lor 0 < t$$
			$$\PA, \ \exists y \ t = s(y) \ \vdash \ 0 < t$$
			
			The first of these is the most straightforward. Call this proof $\mathcal{D}_{1}$.
			
			\begin{center}
				$\begin{array}{c}
					\infer[\to I, 1]{t = 0 \to (t = 0) \lor (0 < t)}
						{\infer[\lor I_{R}]{(t = 0) \lor (0 < t)}
							{\infer[1]{t = 0}
								{}}}
				\end{array}$
			\end{center}
			
			The second subgoal requires an exists elimination. Call this proof $\mathcal{D}_{2}$.
			
			\begin{center}
				$\begin{array}{c}
					\infer[\to I,2]{\exists y \ t = s(y) \to (t = 0 \lor 0 < t)}
						{\infer[\lor I{L}]{(t = 0 \lor 0 < t)}
							{\infer[\exists E^{!!!}]{0 < t}
								{\temp{2}{\exists y \ t = s(y)}
								&
								\infer[\to I,1]{t = s(a) \to 0 < t}
									{\infer[SUGAR]{0 < t}
										{\infer[=E]{t = 0 + s(a)}
											{\temp{1}{t = s(a)}
											&
											\infer[\THM]{0 + s(a) = s(a)}
												{}}}}}}}
				\end{array}$
			\end{center}
			
			$^{!!!}$ Notice that the variable we are doing exists elimination on is $a$. So the fact that $t$ appears free in both antecedent and consquent here does not preclude one from using exists-elimination.
			
			Finally the cases can be combined with $\lor$E and the proofs $\mathcal{D}_{1}$ and $\mathcal{D}_{2}$: 
			
			\begin{center}
				$\begin{array}{c}
					\infer[\forall I]{\forall x \ x = 0 \lor 0 < x}
						{\infer[\lor E]{t = 0 \lor 0 < t}
							{\infer[\THM]{t = 0 \lor \exists y \ t = s(y)}{}
							&
							\mathcal{D}_{1}
							&
							\mathcal{D}_{2}}}
				\end{array}$
			\end{center}
			
			
			\item $\PA \ \vdash \ \forall x \ \forall y \ [\lnot(x<y) \to ((x=y) \lor (y<x))] \hfill \text{(Challenge!)}$
			
			
			\item $\PA \ \vdash \ \forall x \ \forall y \ [(x\leq y) \land (y\leq x)] \to x=y \hfill \text{(Challenge!)}$
		\end{enumerate}
	 
\end{enumerate}	
\end{document}