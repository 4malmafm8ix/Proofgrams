\documentclass[11pt]{report}

% Document dimensions
\usepackage{geometry}
\geometry{top=1.5cm, bottom=1.5cm, textwidth=15cm}

% Math related packges.
\usepackage{amsmath}
\usepackage{cancel}

% Natural Deduction package
\usepackage{proof}

% Fix the header space: start at the top of the page.
\usepackage{hyperref}


\begin{document}
	
	
% Heading for the tutorial	
\begin{center}
	{\bf MATH230: Tutorial Three}
\end{center}
\begin{center}
	{\bf Natural Deductions with Classical Logic}
\end{center}


% Box with goals and relevent lecture notes.
\noindent\fbox{
	\parbox{\textwidth}{

		Key ideas
			\begin{itemize}
				\item Write natural deductions using RAA.
				\item Prove LEM and DNE.
				\item Use LEM and DNE as derived rules of inference to avoid RAA directly.
				\item Witness the oddities of classical theorems. 
			\end{itemize}

		Relevant Topic: Propositional Logic\\
		Relevant reading: \href{https://leanprover.github.io/logic_and_proof/index.html}{L$\exists\forall$N Chapter 5}
		
	\vspace{0.2cm}

	Hand in exercises: 2a, 2b, 3a, 3b, 3e\\ 
	{\bf Due Friday @ 5pm to the submission box on Learn.}
	}
}
% Discussion questions for tutor.
\newline
\vspace{0.5cm}

\noindent {\bf Discussion Questions}

\begin{enumerate}
	
	\item $\lnot (A \land B) \vdash \lnot A \lor \lnot B$
	
\end{enumerate}

% New page for tutorial exercises.
\newpage
{\bf Tutorial Exercises}
\begin{enumerate}

	\item \textbf{NOTE!} Make sure you have finished all of the minimal and intuitionistic natural deductions before doing this tutorial. It is more important that you understand those.
	
	\item \textbf{LEM and DNE.} Prove each of the following fundamental theorems of classical logic making explicit use of the RAA mode of reasoning: 
	
	\begin{enumerate}
		\item $\vdash A \lor \neg A$ \hfill [Challenge!]
		\item $\neg\neg A \vdash A$ 
	\end{enumerate}
	
	\item \textbf{Classical derivations.} Provide natural deduction proofs of the following. All rules \emph{may} be required. Rather than making explicit use of RAA, it can be easier to appeal to LEM or DNE as \emph{derived rules of inference}. 
	
	\begin{enumerate}
		\item $\neg( A \land  B) \vdash  \neg  A \lor \neg  B$ 
		\item $ A \to  B \vdash  \neg  A \lor  B$ 
		\item $\vdash  ( A \to  B) \lor ( B \to  C)$ 
		\item \(\vdash   A\to B\lor\neg B\)  
		\item $\vdash (\neg A\to A)\to A$ 
		\item $( A \to  B) \vdash ( A \rightarrow  D) \lor ( C \rightarrow  B)$ 
		\item $\lnot ( A \to  B) \vdash  A \land \lnot B$ 
		\item $\vdash \ ((A \to B) \to A) \to A$ \hfill [Challenge!]
	\end{enumerate}
	
	\vfill
	\hfill \textbf{PTO}
	\newpage
	
	\item In class we discussed how classical logic can be obtained from intuitionistic logic by adding the following rule of inference \emph{reductio ad absurdum}: If $^{\Sigma}_{\bot}\mathcal{D}$ is a deduction of $\bot$ from $\Sigma$, then
	
	\begin{center}		
		$\begin{array}{c}		
		\infer[\text{RAA}]{\alpha}
		{\begin{array}{c} \hline \cancel{\lnot\alpha} \\ \Sigma \\ \mathcal{D} \\ \bot \end{array}}
		\end{array}$
	\end{center}
	
	is a derivation of $\alpha$ from the assumptions $\Sigma \backslash\{\lnot\alpha\}$.

	In this question we will explore this extension of logics in more detail. We will see that there are different methods for obtaining classical logic from minimal/intuitionistic logic.	For the purposes of this question we introduce two more rules of inference: 

	\begin{center}
		\begin{minipage}{0.45\textwidth}
		\centering

			\textbf{Double Negation Elimination}

			$\begin{array}{ c }
			\infer[\text{DNE}]{\alpha}
			{
			\begin{array}{c} \  \\ \ \\ \lnot \lnot\alpha \to \alpha \end{array}
			& 
			\begin{array}{c} \Sigma  \\ \mathcal{D} \\ \lnot \lnot\alpha \end{array}							 
			}
			\end{array}$
		\end{minipage}%
	\hfill
		\begin{minipage}{0.45\textwidth}
		\centering

			\textbf{Law of Excluded Middle}

			$\begin{array}{ c }
			\infer[\text{LEM}]{\gamma}
			{
			\begin{array}{c} \  \\ \ \\ \alpha\lor\lnot\alpha \end{array}
			& 
			\begin{array}{c} \Sigma_{1}  \\ \mathcal{D}_{1} \\ \alpha\rightarrow\gamma \end{array}				
			& 
			\begin{array}{c} \Sigma_{2}  \\ \mathcal{D}_{2} \\ \lnot\alpha\rightarrow\gamma \end{array}				 
			}
			\end{array}$
		\end{minipage}
	\end{center}

	\textbf{Question:} Suppose you are given a proof witnessing the sequent $\Sigma, \lnot \alpha \ \vdash \ \bot$. 

	\begin{itemize}
		\item[(a)] Using only minimal logic + DNE extend this proof to a proof witnessing the sequent $ \Sigma \ \vdash \alpha$. 
		\item[(b)] Using only inituitionistic logic + LEM extend this proof to a proof witnessing the sequent $ \Sigma \ \vdash \alpha$. 
	\end{itemize}

	Use part (a) to argue that RAA = DNE in the presence of minimal logic. Where as part (b) shows RAA = LEM in the presence of intuitionistic logic. Finally, argue that all three are therefore equivalent modes of reasoning in the presence of intuitionistic logic: RAA = DNE = LEM.		

	\end{enumerate}	
\end{document}