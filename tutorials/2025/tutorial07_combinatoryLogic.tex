\documentclass[11pt]{report}

% Document dimensions
\usepackage{geometry}
\geometry{top=1.5cm, bottom=1.5cm, textwidth=15cm}

% Math related packges.
\usepackage{amsmath}
\usepackage{cancel}

% Natural Deduction package
\usepackage{proof}
\usepackage{mdframed}

% Fix the header space: start at the top of the page.
\usepackage{hyperref}

% Import the necessary preamble for the document. 
\usepackage{../../proofsPrograms}


\begin{document}

% Heading for the tutorial	
\begin{center}
	{\bf MATH230: Tutorial Seven}
\end{center}
\begin{center}
	{\bf Recursion and Combinatory Logic}
\end{center}


% Box with goals and relevent lecture notes.
\noindent\fbox{
	\parbox{\textwidth}{

		Key ideas
			\begin{itemize}
				\item Write recursive processes in $\lambda$-calculus,
				\item Write higher order procedures in $\lambda$-calculus, 
				\item Prove extensional identities in combinatory logic, 
				\item Translate between $\lambda$-calculus and combinatory logic.
			\end{itemize}

		Relevant topic: Untyped Lambda Calculus Slides\\
		Relevant reading: Type Theory and Functional Programming, Simon Thompson
		
	\vspace{0.2cm}

	Hand in exercises: 1b, 4c, 5c, 6a, 7c\\ 
	{\bf Due Friday @ 5pm to the submission box on Learn.}
	}
}
% Discussion questions for tutor.
\newline
\vspace{5mm}

\noindent {\bf Discussion Questions}

\begin{itemize}
	\item Determine some steps towards writing a program ($\lambda$-term) representing the unary function, INT-SQRT, that returns the greatest natural number whose square is less than or equal to the input. 

\end{itemize}

% New page for tutorial exercises.
\newpage
{\bf Tutorial Exercises}

\begin{enumerate}
	
	\item Write recursive $\lambda$-expressions that represent the following functions of natural numbers. For each function determine an appropriate helper-function GO to put through the $\YCOMB$ combinator. 
	
		\begin{enumerate}
			\item SUM of two natural numbers
			\item MULTiply two natural numbers
			\item EXPONentiation of a base to an exponent
			\item FACTorial of a natural number
			\item INT-SQRT the smallest integer whose square is greater than input
			\item Calculate the nth FIBonacci number (Challenge!)
		\end{enumerate}

	\item Write a $\lambda$-expression that can be used to compute the smallest natural number that satisfies a given unary-predicate $P?(x)$ that is represented by some $\lambda$-expression. 		

	\item (Challenge!) Represent the following processes in the $\lambda$-calculus to get an expression that can be used to test whether a natural number is prime. For simplicity, assume the input is greater than TWO.
	
		\begin{enumerate}
			\item REMAINDER calculate the remainder of a division.
			\item DIVIDES? binary predicate does second divide first?
			\item Implement bounded-search to satisfy a predicate.
			\item PRIME? Unary-predicate to detect primality.
		\end{enumerate}
	

	
	 \item In lectures we introduced a $\lambda$-term for computing the sum of a sequence of consecutive integers. This used the helper-function: 
	
	 \begin{align*}
	 	\GO :\equiv \lambda s. \ \lambda a. \ \lambda l. \ \lambda u. \ \COND & \ (>? \ l \ u) \\
	 	& a \\ 
	 	&(s \ (\SUM \ a \ l) \ (\SUCC \ l) \ u)
	 \end{align*}
	
	 We defined ACCUMULATE = $\YCOMB$ GO. Make alterations to the helper-function to compute the following: 
	
	 \begin{enumerate}
	 	\item Compute the sum of the squares of each integer, $\sum_{i=l}^u i^{2}$
	 	\item Compute the sum of each term passed through an arbitrary function, $\sum_{i=l}^u f(i)$
	 	\item Compute the sum of those terms in the interval that satisfy some predicate $P?(x)$.
	 \end{enumerate}
	 
	 \vfill
	 \hfill \textbf{PTO}



\newpage

% Give the reduction definition of the combinators to be used in this tutorial. 
Recall the following reduction rules of the CL combinators. 

\begin{tabular}{l l}
$\textbf{S}xyz \to_{\beta} xz(yz)$ & $\textbf{K}xy \to_{\beta} x$ \\
$\textbf{I}x \to_{\beta} x$ & $\textbf{B}fgx \to_{\beta}f(gx)$ \\
$\textbf{W}fx \to_{\beta}fxx$ & \
\end{tabular}

\item Verify each of the following extensional equality claims by evaluating each side at an appropriate number of variables and check the reductions are identical.

	\begin{enumerate}
		\item \textbf{I} = \textbf{SKK}
		\item \textbf{SK} = \textbf{KI}
		\item \textbf{B} = \textbf{S(KS)K}
		\item \textbf{W} = \textbf{SS(KI)}
	\end{enumerate}

\item Each of these CL terms are reducible. If they have a normal form, then compute it. Otherwise, show that the term has no normal form.

	\begin{enumerate}
		\item \textbf{SKI(KIS)}
		\item \textbf{KS(I(SKSI))}
		\item \textbf{SKIK}	
		\item \textbf{SII(SII)}
	\end{enumerate}

\item Translate each of these $\lambda$-terms into combinatory logic expressions involving only \textbf{SKI} combinators (and free variables) using the translation defined in the lecture slides.

	\begin{enumerate}
		\item $\lambda x. \ \lambda y. \ y$
		\item $\lambda x. \ x \ x$
		\item $(\lambda x. \ x \ x) \ (\lambda x. \ x \ x)$
		\item $\lambda u . \ \lambda v. \ u \ v$
		\item $\lambda x. \ f \ (x \ x)$
		\item $\lambda f. \ \textbf{S}(\textbf{K}f)(\textbf{SII})$
		\item $\lambda f. \ (\lambda x. \ f \ (x \ x)) \ (\lambda x. \ f \ (x \ x))$ 	
	\end{enumerate}

	

	 
\end{enumerate}
	
\end{document}