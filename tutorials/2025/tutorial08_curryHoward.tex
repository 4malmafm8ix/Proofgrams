\documentclass[11pt]{report}

% Document dimensions
\usepackage{geometry}
\geometry{top=1.5cm, bottom=1.5cm, textwidth=15cm}

% Math related packges.
\usepackage{amsmath}
\usepackage{cancel}

% Natural Deduction package
\usepackage{proof}
\usepackage{mdframed}

% Fix the header space: start at the top of the page.
\usepackage{hyperref}

% Import the necessary preamble for the document. 
\usepackage{../../proofsPrograms}


\begin{document}

% Heading for the tutorial	
\begin{center}
	{\bf MATH230: Tutorial Eight}
\end{center}
\begin{center}
	{\bf Curry-Howard Correspondence}
\end{center}


% Box with goals and relevent lecture notes.
\noindent\fbox{
	\parbox{\textwidth}{

		Key ideas
			\begin{itemize}
				\item Write context dependent typing derivations.
				\item Understand the connection between natural deductions and typing derivations.
				\item Write proof-terms witnessing theorems of minimal logic.
			\end{itemize}

		Relevant lectures: Typed Lambda Calculus Slides\\
		Relevant reading: Type Theory and Functional Programming, Simon Thompson
		
	\vspace{0.2cm}

	Hand in exercises: 1a, 1d, 1e, 1k, 3\\ 
	{\bf Due Friday @ 5pm to the submission box on Learn.}
	}
}
% Discussion questions for tutor.
\newline
\vspace{0.5cm}

\noindent {\bf Discussion Questions}

\begin{itemize}
	\item Write a program of the specified type in the given context: 
	
	$$p : A \times (B \times C) \ \vdash \ (A \times B) \times C$$
	
	\vspace{50mm}
	
		\item  For a fixed typed $A$, prove that the type $(A \to A) \to A$ is uninhabited i.e. there is no term $t$ of simple type theory that has this type. 

\end{itemize}

% New page for tutorial exercises.
\newpage
{\bf Tutorial Exercises}

\begin{enumerate}
	\item For each $\Sigma \ \vdash \ \alpha$ provide a term of type $\alpha$ from the given $\Sigma$ context. 
	\begin{enumerate}
		\item $f : A \to (B \to C) \ \vdash \ B \to (A \to C)$ \hfill [Handin exercise]
		\item $t : A \times B \ \vdash \ B \times A$
		\item $t : A + B \ \vdash \ B + A$ 
		\item $f : (A\times  B) \rightarrow  C \vdash  A\rightarrow ( B \rightarrow  C) $ \hfill [Handin exercise]
		\item $f : A \rightarrow ( B \rightarrow  C) \vdash ( A\times  B) \rightarrow C$	 \hfill [Handin exercise]
		\item $f : A \to B \ \vdash A \to (B + C)$
		\item $ f : A\rightarrow  B, \  g : B \rightarrow  C \vdash  A\rightarrow  C $
		\item $t : A +  B, \  f : A\to  C, \  g : B \to  D \vdash   C +  D$		
		\item $f : A \to B \ \vdash \ (C \to A) \to (C \to B)$	
		\item $t : (A \to B) \times (A \to C) \ \vdash \ A \to (B \times C)$
		\item $t : A \times (B + C) \ \vdash \ (A \times B) + (A \times C)$ \hfill [Handin exercise]
		\item $t : (A \times B) + (A \times C) \ \vdash \ A \times (B + C)$
		\item $t : A + (B \times C) \ \vdash \ (A + B) \times (A + C)$
		\item $t : (A + B) \times (A + C) \ \vdash \ A + (B \times C)$
	\end{enumerate}
	
		{\bf Extras:} For these extra problems consider $\bot$ to be type with no constructor or destructors. Furthermore, consider $\lnot P$ to be shorthand for the function type: $\lnot P:= P \to \bot$.

		\begin{enumerate}
			\item $f : \lnot A \ \vdash \ (C \to A) \to \lnot C$
			\item $t : A \times \lnot B \ \vdash \ \lnot (A \to B)$
			\item $f : A \to  C, \  g : B \to  D, \ t : \neg C + \neg  D \vdash  \neg A+ \neg  B$
			\item $ t : A, \ f : \neg  A\vdash  \neg  B$
			\item $ f : A\rightarrow B, \  g : A\rightarrow \lnot B \vdash \lnot  A$
			\item $f : A \to \lnot B \ \vdash \ B \to \lnot A$
			\item $f : \lnot (A \times B) \ \vdash \ A \to \lnot B$
			\item $t : A \ \vdash \ \lnot\lnot A$
			\item $f : \lnot \lnot \lnot A \ \vdash \ \lnot A$
			\item $t : \lnot A + \lnot B \vdash \lnot( A\times  B)$
			\item $f : \lnot  A\times \lnot  B \ \vdash \lnot( A +  B)$
			\item $f : \lnot( A +  B) \ \vdash \lnot  A\times \lnot  B$
			\item $f : A \to \lnot B \ \vdash \ \lnot (A \times B)$
			\item $\vdash \ \lnot\lnot (A + \lnot A)$
		\end{enumerate}

	\item Revisit Lab 1 and Lab 2. For each derivation in those labs, provide a proof-object witnessing a natural deduction of the sequent. You don't need to do any more derivations at this point!

	\newpage
	\item This exercise shows you an example of a general observation first made by William Tait, relating the simplifications of proofs and the process of computation in the $\lambda$-calculus. 
	
	Consider the following proof of the theorem $$\vdash \ A \land B \to B$$
	
	\begin{center}
		$\begin{array}{c}		
		  \infer[\to, 1]{A \land B \to B}
		  	{\infer[\land E_{L}]{B}
				{\infer[\land I]{B \land A}
					{\infer[\land E_{R}]{B}{\infer[1]{A \land B}{}} 
					\hspace{0.5cm}	&	\hspace{0.5cm}
					\infer[\land_{L}]{A}{\infer[1]{A \land B}{}}}}}
		\end{array}$
	  \end{center}

	  	\begin{enumerate}
			\item Determine the corresponding proof-object for this proof. 
			\item Why does the proof-object have a redex in it? 
			\item Perform the $\beta$-reduction on the proof object from (a).
			\item What proof does the reduced proof-object correspond to?
		\end{enumerate} 

	\item Prove that the type $A + B \to A$ is uninhabited i.e. there is no term $t$ of simple type theory that has this type. Your proof should be an informal reason for why no such program can exist. You might refer to the corresponding minimal logic sequent to help your justification.	

	 
\end{enumerate}
	
\end{document}