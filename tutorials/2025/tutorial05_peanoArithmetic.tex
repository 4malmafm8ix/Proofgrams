\documentclass[11pt]{report}

% Document dimensions
\usepackage{geometry}
\geometry{top=1.5cm, bottom=1.5cm, textwidth=15cm}

\newcommand{\prop}{\text{Prop}}

% Rule of inference shortcuts
\newcommand{\IND}{\text{IND}}
\newcommand{\THM}{\text{THM}}
\newcommand{\MP}{\text{MP}}
\newcommand{\PA}{\text{PA}}
\newcommand{\IH}{\text{IH}}
\newcommand{\RAA}{\text{RAA}}
\newcommand{\DNE}{\text{DNE}}
\newcommand{\DM}{\text{DM}}
\newcommand{\XF}{\text{XF}}
\newcommand{\MT}{\text{MT}}

% Math related packges.
\usepackage{amsmath}
\usepackage{cancel}

% Natural Deduction package
\usepackage{proof}
\usepackage{mdframed}

% Fix the header space: start at the top of the page.
\usepackage{hyperref}


\begin{document}
	
	
% Heading for the tutorial	
\begin{center}
	{\bf MATH230: Tutorial Five}
\end{center}
\begin{center}
	{\bf Peano Arithmetic}
\end{center}


% Box with goals and relevent lecture notes.
\noindent\fbox{
	\parbox{\textwidth}{

		Key ideas
			\begin{itemize}
				\item Natural deduction practice, 
				\item Proofs using the identity rules of inference,
				\item Prove first-order sentences in theories of arithmetic,
				\item Use the induction schema of Peano arithmetic, and 
				\item Become exasperated enough to appreciate the help of proof assistants. 
			\end{itemize}

		Relevant Topic: Peano Arithmetic\\
		Relevant reading: \href{https://adam.math.hhu.de/\#/g/leanprover-community/nng4}{Natural Number Game}
		
	\vspace{0.2cm}

	Hand in exercises: 1c, 2a, 2b, 2c, and 5b\\ 
	{\bf Due Friday @ 5pm to the submission box on Learn.}
	}
}
% Discussion questions for tutor.
\newline
\vspace{0.5cm}

\noindent {\bf Discussion Questions}

	Proofs that make use of the axiom (schema) PA 7 

	$$[P(0) \land \forall x \ (P(x) \to P(s(x)))] \rightarrow \forall y (P(y))$$

	will prove statements of the form $\forall y \ P(y)$ with the use of modus ponens. This requires proving the antecedent conjunction: 

	$$[P(0) \land \forall x \ (P(x) \to P(s(x)))]$$

	This in turn requires proving each conjunct i.e. two proofs witnessing: 

	$$\PA \vdash P(0) \hspace{2cm} \PA \ \vdash \ \forall x \ (P(x) \to P(s(x)))$$

	If we piece these together, then we see that all proofs by induction have the form: 

	\begin{mdframed}
		\begin{center}
			$\begin{array}{c}
				\infer[\IND]{\forall y \ P(y)}
					{\infer[\land I]{[P(0) \land \forall x \ (P(x) \to P(s(x)))]}
						{\infer[]{P(0)}{\begin{array}{c} \vdots \\ \mathcal{D}_{BC}\end{array}}
						&
						\infer[\forall I]{\forall x \ (P(x) \to P(s(x)))}
							{\begin{array}{c} \vdots \\ \mathcal{D}_{IS}\end{array}}}}
			\end{array}$
		\end{center}
	\end{mdframed}

	\begin{enumerate}
		\item Identify the following steps involved in a proof by induction of the following sentence of Peano arithmetic: 
		
		$$\PA \ \vdash \ \forall x \ (0+x=x) $$

			\begin{enumerate}
				\item Identify the wff $P(x)$ to do induction on. 
				\item $\mathcal{D}_{BC}$: Write down the sequent $\PA \ \vdash \ P(0)$.
				\item $\mathcal{D}_{IS}$: Write down the sequent $\PA, P(n) \ \vdash \ P(s(n))$.	
			\end{enumerate}
	\end{enumerate}

% New page for tutorial exercises.
\newpage
{\bf Tutorial Exercises}
\begin{enumerate}

	\item Give natural deductions of the following theorems of identity.
	
	\begin{enumerate}
		\item $\vdash \ \exists x \ (t = x)$
		\item $\vdash \ \forall x  \forall y \ x = y \to y = x$
		\item $\vdash \ \forall x \forall y \forall z \ (x = y \land y = z) \to x = z$ 
		\item $\vdash \ \forall x \forall y \forall z \ x \neq y \to (x \neq z \lor y \neq z)$ \hspace{5cm} (RAA)

	\end{enumerate}

	Parts (b) and (c) together with proofs from lectures show that identity is reflexive, symmetric, and transitive. Thus behaving like an equivalence relation - as one would hope of the definition of equals!
	
	\item In this question $\text{PA}$ denotes the first-order theory of Peano arithmetic which has signature PA$: \{0,s,+,\times\}$ and axioms:
	
	\begin{enumerate}
		\item[PA1] $\forall x \lnot(s(x) = 0)$
		\item[PA2] $\forall x \ \forall y ((s(x) = s(y)) \to (x = y))$
		\item[PA3] $\forall x \ (x + 0 = x)$
		\item[PA4] $\forall x \ \forall y \ (x + s(y) = s(x + y))$
		\item[PA5] $\forall x \ (x \times 0 = 0)$
		\item[PA6] $\forall x \ \forall y \ (x \times s(y) = (x \times y) + x)$
		\item[PA7] $[P(0) \land \forall x \ (P(x) \to P(s(x)))] \rightarrow \forall y (P(y))$
	\end{enumerate}	
	
	Provide deductions to prove the following sequents. 
	
	\begin{enumerate}
		\item $\text{PA} \vdash 1 + 1 = 2$
		\item $\text{PA} \vdash 3 \neq 1$
		\item $\text{PA} \vdash \forall x \ (x + 1 = s(x))$
		\item $\text{PA} \vdash \forall x \ (x \times 1 = x)$ 		
	\end{enumerate}

		\item The first-order language of Peano Arithmetic is often presented with an extra binary relation symbol $<$ where $x < y$ is given the usual interpretation: $x$ is \emph{strictly} less than $y$. In fact it is not necessary to add anything extra, for this relation can be defined using a sentence in $\PA$ as stated. 
	
	Write down a wff in $\PA$ which defines the binary relation $<$ of being ``strictly less than''. Use this to write down formulae that represent: less than or equal to, strictly greater than, and greater than or equal to. 

	%$$ x < y :\equiv \exists z \ (x + s(z) = y) $$

	\item Write down well-formed formulae in the first-order language of $\PA$ corresponding to the following statements. 
	
	\begin{enumerate}
		\item Each natural number is either equal to $0$ or greater than $0$. 			
		\item If $x$ is not less than $y$, then $x$ equals $y$ or $y$ is less than $x$.
		\item If $x$ is less than or equal to $y$ and $y$ is less than or equal to $x$, then $x=y$. 
	\end{enumerate} 

	\newpage
	\item The followings sequents all require the use of the induction axiom schema. Recall that all proofs using the induction schema have the following form: 
	
		\begin{mdframed}
			\begin{center}
				$\begin{array}{c}
					\infer[IND]{\forall y \ P(y)}
						{\infer[\land I]{[P(0) \land \forall x \ (P(x) \to P(s(x)))]}
							{\infer[]{P(0)}{\begin{array}{c} \vdots \\ \mathcal{D}_{BC}\end{array}}
							&
							\infer[]{\forall x \ (P(x) \to P(s(x)))}
								{\begin{array}{c} \vdots \\ \mathcal{D}_{IS}\end{array}}}}
				\end{array}$
			\end{center}
		\end{mdframed}

	For this reason, once the wff $P(x)$ is identified, it suffices to provide the base case deduction $\mathcal{D}_{BC}$ and induction step $\mathcal{D}_{IS}$. The sequents are stated in such a way as to mean induction on the variable $x$ will be the easiest approach. Always do induction on the variable $x$.  
	
	\begin{enumerate}
		\item $\text{PA} \vdash \forall x \ (0 + x = x)$
		\item $\text{PA} \vdash \forall x \ (0 \times x = 0)$
		\item $\text{PA} \vdash \forall x \ (1 \times x = x)$
		\item $\text{PA} \vdash \forall x \ (x = 0 \lor \exists y (x = s(y)))$ \hfill (Challenge!)
		\item $\text{PA} \vdash \forall x \ \forall y \ [s(y) + x = s(y+x)]$ \hfill (Challenge!)
		\item $\text{PA} \vdash \forall x \ \forall y \ \forall z \ [(y + z) + x = y + (z + x)]$ \hfill (Challenge!)
		\item $\text{PA} \vdash \forall x \ \forall y \ [y + x = x + y]$ \hfill (Challenge!)
	\end{enumerate}

	\item Visit the \href{https://adam.math.hhu.de/\#/g/leanprover-community/nng4}{Natural Number Game} to write computer checked proofs of these theorems of Peano arithmetic as well as theorems of propositional logic from previous tutorials. 
	 
	\item Provide natural deductions of the following theorems of Peano Arithmetic. Beware each of these must be true, but some I have not provided natural deductions for. Some may require breaking down into further subgoals (lemma) to help. I recommend writing informal proofs, before formalising them with natural deductions.
	
		\begin{enumerate}
			\item $\PA , \ 0 < a \ \vdash \ 0 < s(a)$
			\item $\PA \ \vdash \ a < s(a)$
			\item $\PA , \ a < b \ \vdash \ s(a) < s(b)$
			\item $\PA , \ s(a) < s(b) \ \vdash \ a < b$
			\item $\PA , \ (a < b) \land (b < c) \ \vdash \ a < c$
			\item $\PA \ \vdash \ \forall x \ [(x = 0) \lor (0 < x)] \hfill \text{(Challenge!)}$
			\item $\PA \ \vdash \ \forall x \ \forall y \ [\lnot(x<y) \to ((x=y) \lor (y<x))] \hfill \text{(Challenge!)}$
			\item $\PA \ \vdash \ \forall x \ \forall y \ [(x\leq y) \land (y\leq x)] \to x=y \hfill \text{(Challenge!)}$
		\end{enumerate}
	 
\end{enumerate}	
\end{document}